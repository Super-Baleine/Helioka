\documentclass[a4paper,10pt]{article}
\usepackage[utf8]{inputenc}

\title{Helioka - Telecommunication Protocol over the Internet}
\author{Paul Feuvraux}

\begin{document}

\maketitle

\section{Introduction}
This paper aims to explains Helioka, a simple protocol based on the RSA and AES algorithms only.

\section{Motivations}
For fun.

\section{Terms}
\begin{itemize}
    \item \textbf{Key Agreement Key ($KAK$)}
    4096-bit asymmetric RSA keys ($PKA$ and $PKB$) generated at the client-side.
    \item \textbf{Private Key ($PKA$)}:
    Used to decrypt the $TEK$.
    \item \textbf{Public Key ($PKB$)}:
    Used to encrypt the $TEK$.
    \item \textbf{Temporary Encryption Key ($TEK$)}:
    256-bit symmetric AES key generated at the client-side. Used to encrypt the communications.
    \item \textbf{Content Encryption Key ($CEK$)}:
    256-bit symmetric AES key generated at the client-side. Used to encrypt $PKA$.
    \item \textbf{Key Encryption Key ($KEK$)}:
    256-bit symmetric AES key derived from the user's Passphrase $P$.
    \item \textbf{Passphrase ($P$)}:
    User's defined alphanumeric UTF-8 passphrase during registration on the client side.
    \item \textbf{Key Derivation function}:
    Every derivation function is performed with PBKDF2. We denote this process as $KDF(x, s, i)$, where x is the passphrase, s the Salt, and i is the strengthen by factor and always equals 7000.
    \item \textbf{Symmetric Encryption function}:
    Every symmetric encryption is performed with AES under the Galois/Counter mode on 128-bit block cipher. We denote this process as $EncSym(k, x, i, a)$, where $k$ is a symmetric AES key, x the content to be encrypted, i is the initialization vector (IV) and a is the authentication data (AD).
    \item \textbf{Symmetric Decryption function}:
    Every symmetric decryption is performed with AES under the Galois/Counter mode on 128-bit block cipher. We denote this process as $DecSym(k,x,i,a)$, where k is a symmetric AES key, x the encrypted content to be decrypted, i is the IV and a is the authentication data (AD).
    \item \textbf{Asymmetric Encryption function}:
    Every asymmetric encryption function is performed with RSA. We denote this process as $EncAsym(k,x)$, where k is a Public Key $PKB$ and x is the content to be encrypted.
    \item \textbf{Asymmetric Decryption function}:
    Every asymmetric decryption is performed with RSA. We denote this process as $DecAsym(k,x)$, where k is a Private Key $PKA$ and x is the encrypted content to be decrypted.
    
\end{itemize}

\section{Protocol}
\subsection{Registration}
While the user is typing his Passphrase $P$, the necessary keys ($KAK$ and $CEK$) are generated.
\subsubsection{Symmetric key encryption}
A random Salt is generated on 128 bits. Once the Passphrase $P$ defined by the user, the client proceeds to a key derivation such as $KEK=KDF(P,Salt, 7000)$. Once the $KEK$ obtained through this key derivation step, a random IV and AD are both generated on 128 bits. Then, the client encrypts the $CEK$ under the $KEK$ such as $EncSym(KEK, CEK,IV,AD)$ and sends it to the server as a packet such as $packet=(CEK||IV||AD||Salt)$, where $CEK$ is the encrypted $CEK$.
\subsubsection{Asymmetric key encryption}
The same $CEK$ is used to encrypt $PKA$. The client sends $PKB$ in plain text to the server while a random IV and AD are both generated on 128 bits. Once these parameters generated, the client encrypts $PKA$ under $CEK$ such as $EncSym(CEK, PKA, IV, AD)$, where $CEK$ is the plain text $CEK$. Once encrypted, the client sends it as a packet $packet$ such as $packet=(PKA||IV||AD)$, where $PKA$ is the encrypted $PKA$.

\subsection{Connection}
The user types his Passphrase $P$.

\subsubsection{$CEK$ decryption}
The client gets the $CEK$ and extract the Salt from it. Once the Salt extracted, the client proceeds to a Key Derivation such as $KEK=KDF(P,Salt,7000)$. Once the $KEK$ derived, the clients extract the IV and the AD from the $CEK$ and decrypts the $CEK$ such as $DecSym(KEK,CEK,IV,AD)$, where $CEK$ is the encrypted $CEK$.
\subsubsection{$PKA$ decryption}
The client gets the $KAK$ and extract the $PKA$ from it, which is a packet composed of the encrypted $PKA$ and its cryptographic parameters. The client decrypts $PKA$ under $CEK$ such as $DecSym(CEK, PKA, IV,AD)$.

\subsection{$TEK$ exchange}
One of the two participants' client generates the $TEK$ and an IV and an AD, both on 128 bits. The client makes a packet with the $TEK$, the IV, and the AD such as $packet=(TEK||IV||AD)$. The client gets the other user's $PKB$ and encrypts $packet$ such as $EncAsym(PKB,packet)$ and sends it.

\subsection{Communication Encryption}
Every "song" is turned into small packets. Every packet is encrypted under $TEK$.

\subsubsection{Voice splitting}
Every 20ms the voice is turned into a 256-bit block $B_y$, where $y$ is the ID of the block. The block $B_y$ is split in eight packets $Z$ such as $Z_x=B_y/8$, where $x$ is the ID of the packet. 

\subsubsection{Encryption of $Z$}
For each packet $Z_x$ the client encrypts it such as $EncSym(TEK,Z_x,IV,AD)$. Each packet $Z_x$ is sent once encrypted encapsulated with the ID of the block such as $Z=(Z_x||y)$ and sends it.

\subsubsection{Decryption of $Z$}
For every packet $Z$, the receiver's client extracts $Z_x$ and $y$ from $Z$ and decrypts $Z_x$ such as $DecSym(TEK,Z_x,Iv,AD)$. Once every packet decrypted, the clients reassembles them such as $B_y=(Z_1||Z_2||Z_3||Z_4||Z_5||Z_6||Z_7||Z_8)$.

\subsection{User's Passphrase change}
The user defines his new Passphrase $P$. A random Salt is generated on 128 bits, the client proceeds to a Key Derivation such as $KEK=KDF(P, Salt, 7000)$. Once the new $KEK$ obtained from the Key Derivation function, a random IV and AD are both generated on 128 bits. Then, the clients encrypts the $CEK$ under the new $KEK$ derived from the new $P$ such as $EncSym(KEK,CEK,IV,AD)$. Once the $CEK$ encrypted, the client send the encrypted $CEK$ to the server as a packet $packet$ such as $packet=(CEK||IV||AD||Salt)$, where $CEK$ is the encrypted $CEK$.
 
\end{document}

